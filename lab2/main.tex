\documentclass{article}

% Language setting
% Replace `english' with e.g. `spanish' to change the document language
\usepackage[T2A]{fontenc}
\usepackage[utf8]{inputenc}

% Set page size and margins
% Replace `letterpaper' with `a4paper' for UK/EU standard size
\usepackage[letterpaper,top=2cm,bottom=2cm,left=3cm,right=3cm,marginparwidth=1.75cm]{geometry}

% Useful packages
\usepackage{amsmath}
\usepackage{graphicx}
\usepackage{float}
\usepackage[colorlinks=true, allcolors=blue]{hyperref}

\title{Лабораторная работа №1}
\author{Выполнили: Цалов В.С. Тахватулин М.В.}

\begin{document}
\maketitle
\begin{center}
      {\fontsize{14}{15}\selectfont
            Преподователь: Оранский С.И.
      }
\end{center}

% \title{Вариант 4}

\section{Задание №1}\label{sec:-no1}
\subsection{Текст задания}
Сгенерируйте 2 выборки объёма 25 и посчитайте доверительный интервал. Повторить 1000 раз. Посчитайте, сколько раз 95-
процентный доверительный интервал покрывает реальное значение параметра. То же самое
сделайте для объема выборки 10 000.

\subsection{Выполнение}
В ходе эксперимента были сгенерированы 1000 выборок из задания. 
После этого найдены их выборочные дисперсии и, с помощью распределения Фишера найден доверительный интервал для отношения дисперсий.
Тоже самое было сделано для выборки размеров в 10 000. 

\subsection{Вывод программы}
вывод:\newline
объем выборки 25:\newline
попавшие - 944, точность - 0.944\newline
объем выборки 10к:\newline
попавшие - 952, точность - 0.952


\subsection{Программа}
\begin{verbatim}
import numpy as np
import scipy.stats as stats

# Параметры задачи
mu1, mu2 = 0, 0
sigma1, sigma2 = 2, 1
alpha = 0.05


def sol(n1, n2, mu1, mu2, sigma1, sigma2, alpha):
    total = 0
    probability = 0
    for i in range(1000):
        # генерация выборок
        X1 = np.random.normal(mu1, np.sqrt(sigma1), n1)
        X2 = np.random.normal(mu2, np.sqrt(sigma2), n2)

        # выборочные дисперсии
        S1 = np.var(X1, ddof=1)
        S2 = np.var(X2, ddof=1)

        tau = S1 / S2
        tau_real = sigma1 / sigma2

        # криты распределения фишера
        F_low = stats.f.ppf(alpha / 2, n1 - 1, n2 - 1)
        F_high = stats.f.ppf(1 - alpha / 2, n1 - 1, n2 - 1)

        # доверительный интервал
        CI_low = tau / F_high
        CI_high = tau / F_low

        if CI_low <= tau_real <= CI_high:
            total += 1

    probability = total / 1000
    return total, probability


total1, probability1 = sol(25, 25, mu1, mu2, sigma1, sigma2, alpha)
print(f"объем выборки 25:")
print(f"попавшие - {total1}, точность - {probability1}")

total1, probability1 = sol(10000, 10000, mu1, mu2, sigma1, sigma2, alpha)
print(f"объем выборки 10к:")
print(f"попавшие - {total1}, точность - {probability1}")
\end{verbatim}

\subsection{Вывод}
Результаты программы подтверждают говорят, что около 5\% всех попыток не попадают в доверительный интервал. Что соответствует теории.

\section{Задание №2. Вариант 4}
\subsection{Текст задания}
Постройте асимптотический доверительный интервал уровня $1 - \alpha$ для указанного параметра. Проведите эксперимент по схеме, аналогичной первой задаче.
Вариант 4:\newline
$Geom(p), p = 0.7$
\subsection{Выполнение}
В ходе эксперимента генерируем две выборки, одну размером 25, другую 10000. После находим оценку параметра p. Затем стандартуню ошибку и критическое значение. После этого находим верхние и нижние границы асимптотических доверительных интервалов и проверяем попадание параметра в них.

\subsection{Вывод программы}
Вывод:\newline
попавшие - 986, точность - 0.986\newline
попавшие - 1000, точность - 1.0

\subsection{Программа}
\begin{verbatim}
import numpy as np
import scipy.stats as stats

p = 0.7
alpha = 0.05
n1 = 25
n2 = 10000
total1 = 0
total2 = 0
probability1 = 0
probability2 = 0

for i in range(1000):
    X1 = np.random.geometric(p, n1)
    X2 = np.random.geometric(p, n2)

    # оценка мат ожидания
    p_hat1 = 1 / np.mean(X1)
    p_hat2 = 1 / np.mean(X2)

    # стандартная ошибка
    se1 = np.sqrt((1 - p_hat1) / (n1 * p_hat1 ** 2))
    se2 = np.sqrt((1 - p_hat2) / (n2 * p_hat2 ** 2))

    crit = stats.norm.ppf(1 - alpha / 2)

    ci_low1 = p_hat1 - crit * se1
    ci_upp1 = p_hat1 + crit * se1

    ci_low2 = p_hat2 - crit * se2
    ci_upp2 = p_hat2 + crit * se2

    if ci_low1 <= p <= ci_upp1:
        total1 += 1

    if ci_low2 <= p <= ci_upp2:
        total2 += 1

    probability1 = total1 / 1000
    probability2 = total2 / 1000


print(f"попавшие - {total1}, точность - {probability1}")
print(f"попавшие - {total2}, точность - {probability2}")
\end{verbatim}

\subsection{Вывод}
Можно сделать вывод, что при увеличении размера выборки до 10.000 вероятность ошибики значительно уменьшается, так как в знаменателе $10.000 ^ 2$. Таким образом из 1000 эксперементов попали все




\end{document}