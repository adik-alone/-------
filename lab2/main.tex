\documentclass{article}

% Language setting
% Replace `english' with e.g. `spanish' to change the document language
\usepackage[T2A]{fontenc}
\usepackage[utf8]{inputenc}

% Set page size and margins
% Replace `letterpaper' with `a4paper' for UK/EU standard size
\usepackage[letterpaper,top=2cm,bottom=2cm,left=3cm,right=3cm,marginparwidth=1.75cm]{geometry}

% Useful packages
\usepackage{amsmath}
\usepackage{graphicx}
\usepackage{float}
\usepackage[colorlinks=true, allcolors=blue]{hyperref}

\title{Лабораторная работа №1}
\author{Выполнили: Цалов В.С. Тахватулин М.В.}

\begin{document}
\maketitle
\begin{center}
      {\fontsize{14}{15}\selectfont
            Преподователь: Оранский С.И.
      }
\end{center}

% \title{Вариант 4}

\section{Задание №1}\label{sec:-no1}
\subsection{Текст задания}
Сгенерируйте 2 выборки объёма 25 и посчитайте доверительный интервал. Повторить 1000 раз. Посчитайте, сколько раз 95-
процентный доверительный интервал покрывает реальное значение параметра. То же самое
сделайте для объема выборки 10000.

\subsection{Выполнение}
Генерируем выборку. 
Много выборок, много много много выборок

\section{Задание №2. Вариант 4}
\subsection{Текст задания}
Постройте асимптотический доверительный интервал уровня $1 - \alpha$ для указанного параметра. Проведите эксперимент по схеме, аналогичной первой задаче.

Вариант 4 = $Geom(p), p = 0.7$
\subsection{Выполнение}




\end{document}