\documentclass{article}

% Language setting
% Replace `english' with e.g. `spanish' to change the document language
\usepackage[T2A]{fontenc}
\usepackage[utf8]{inputenc}

% Set page size and margins
% Replace `letterpaper' with `a4paper' for UK/EU standard size
\usepackage[letterpaper,top=2cm,bottom=2cm,left=3cm,right=3cm,marginparwidth=1.75cm]{geometry}

% Useful packages
\usepackage{amsmath}
\usepackage{graphicx}
\usepackage{float}
\usepackage[colorlinks=true, allcolors=blue]{hyperref}

\title{Лабораторная работа №1}
\author{Выполнили: Цалов В.С. Тахватулин М.В.}

\begin{document}
\maketitle
Преподователь: Оранский С.И.

% \title{Вариант 4}

\section{Задание №1}
\subsection{Выбранное распределение}

В качестве распределения было выбрано распределение хи-квадрат.
Плотность распределения будет здесь может быть

Гистограмма этого распеределения выглядит примерно так:

\begin{center}
      \centering
      \includegraphics[width=0.5\linewidth]{Python/chi.png}
\end{center}

\subsection{Первый эксперимент}
В ходе данного задания были проведен эксперимент в ходе которого были созданы 1000 выборок по 1000 значений данного распеределения.
в этих выборках были подсчитаны: среднее выборочное значение, выборочная дисперсия и выборочный квантиль порядка 0.5 (медиана выборки) 

Гистограммы этих значений
\begin{figure}[H]
      \centering
      \includegraphics[width=0.8\linewidth]{Python/first-exp.png}
      \caption{Гистограммы параметров выборки}
\end{figure}
% \begin{center}
%       \begin{figure}
%             \centering
%             \includegraphics[width=0.5\linewidth]{Python/hist-mean.png}
%             \caption{Среднее выборочное значение для всех выборок}
%       \end{figure}
%       \begin{figure}
%             \centering
%             \includegraphics[width=0.5\linewidth]{Python/hist-mean.png}
%             \caption{Среднее выборочное значение для всех выборок}
%       \end{figure}
%       \begin{figure}
%             \centering
%             \includegraphics[width=0.5\linewidth]{Python/hist-mean.png}
%             \caption{Среднее выборочное значение для всех выборок}
%       \end{figure}
% \end{center}

\subsection{Код}
\begin{verbatim}
      import numpy as np
import matplotlib.pyplot as plt
from scipy.stats import gamma, chi



shape = 2
rate = 1

distrib_func_values = []
values = []
for i in range(1000):
    distribution = chi(4)
    samples = np.sort(distribution.rvs(1000))
    sec_value = samples[1]
    distrib_func_value = (distribution.cdf(sec_value) * 1000)
    values.append(distrib_func_value)
distrib_func_values.append(values)

hist, bins = np.histogram(distrib_func_values, bins=30, density=True)
plt.bar(bins[:-1], hist, width=np.diff(bins), color='skyblue', alpha=0.7, label='nF(X_2)')

x = np.linspace(0, 20, 100)
y = gamma.pdf(x, a=shape, scale=1 / rate)

plt.plot(x, y, color='red', linestyle='--', label='Г(2,1)')
plt.legend()
plt.tight_layout()
plt.savefig('second-two-one.png')
plt.show()




for i in range(1000):
    distribution = chi(4)
    samples = np.sort(distribution.rvs(1000))
    last_value = samples[-1]
    distrib_func_value = ((1 - distribution.cdf(last_value)) * 1000)
    values.append(distrib_func_value)
distrib_func_values.append(values)

hist, bins = np.histogram(distrib_func_values, bins=30, density=True)
plt.bar(bins[:-1], hist, width=np.diff(bins), color='skyblue', alpha=0.7, label='n(1 - F(X_n))')

shape = 1
rate = 1

x = np.linspace(0, 10, 50)
y = gamma.pdf(x, a=shape, scale=1 / rate)

plt.plot(x, y, color='red', linestyle='--', label='Г(1,1)')
plt.legend()
plt.tight_layout()
plt.savefig('second-one-one.png')
plt.show()


\end{verbatim}


\subsection{Сравнение с Гамма-распределением}
Второй эксперимент заключается в том, что нам надо сравнить порядковые статистки с гамма-распределением.
Сортируем каждую выборку для нахождения второй и n-ой порядковой статистики $F(X_{(2)}$ и $F(X_{(n)})$.
Затем находим значения $nF(X_{(2)}$ и $n(1 - F(X_{(n)}))$ и затем заносим их в выборку.
Далее генерируем гамма распределение с параметрами $\Gamma(2, 1)$ и $\Gamma(1, 1)$

В итоге получаем такие графики:

\begin{figure}[H]
      \centering
      \includegraphics[width=0.5\linewidth]{Python/second-two-one.png}
\end{figure}
\begin{figure}[H]
      \centering
      \includegraphics[width=0.5\linewidth]{Python/second-one-one.png}
\end{figure}

\subsection{Код второй части}
\begin{verbatim}
      import numpy as np
import matplotlib.pyplot as plt
from scipy.stats import gamma, chi



shape = 2
rate = 1

distrib_func_values = []
values = []
for i in range(1000):
    distribution = chi(4)
    samples = np.sort(distribution.rvs(1000))
    sec_value = samples[1]
    distrib_func_value = (distribution.cdf(sec_value) * 1000)
    values.append(distrib_func_value)
distrib_func_values.append(values)

hist, bins = np.histogram(distrib_func_values, bins=30, density=True)
plt.bar(bins[:-1], hist, width=np.diff(bins), color='skyblue', alpha=0.7, label='nF(X_2)')

x = np.linspace(0, 20, 100)
y = gamma.pdf(x, a=shape, scale=1 / rate)

plt.plot(x, y, color='red', linestyle='--', label='Г(2,1)')
plt.legend()
plt.tight_layout()
plt.savefig('second-two-one.png')
plt.show()




for i in range(1000):
    distribution = chi(4)
    samples = np.sort(distribution.rvs(1000))
    last_value = samples[-1]
    distrib_func_value = ((1 - distribution.cdf(last_value)) * 1000)
    values.append(distrib_func_value)
distrib_func_values.append(values)

hist, bins = np.histogram(distrib_func_values, bins=30, density=True)
plt.bar(bins[:-1], hist, width=np.diff(bins), color='skyblue', alpha=0.7, label='n(1 - F(X_n))')

shape = 1
rate = 1

x = np.linspace(0, 10, 50)
y = gamma.pdf(x, a=shape, scale=1 / rate)

plt.plot(x, y, color='red', linestyle='--', label='Г(1,1)')
plt.legend()
plt.tight_layout()
plt.savefig('second-one-one.png')
plt.show()


\end{verbatim}

\section{Задание №2. Вариант 4}
\subsection{Выполнение}
Из данных надо было найти количество телефонов с различными параметрами.

Итог:
Количество моделей с двумя симкартами: 1019

Количество моделей поддерживающих 3g: 1523

Наибольшее число ядер у процессора: 8

Далeе надо было посчитать параметры выборки телефонов:

Выборочное среднее емкости аккумулятора: 1238.5185

Выборочная дисперсия емкости аккумулятора: 193088.35983766883

Выборочная медиана емкости аккумулятора: 1226.0

Выборочная квантиль порядка 2/5 емкости аккумулятора: 1076.0

Построить графики.
Эмперические функции:
\begin{figure}[H]
      \centering
      \includegraphics[width=0.5\linewidth]{Python/emper-all-phones.png}
\end{figure}
\begin{figure}[H]
      \centering
      \includegraphics[width=0.5\linewidth]{Python/emper-wi-fi.png}
\end{figure}
\begin{figure}[H]
      \centering
      \includegraphics[width=0.5\linewidth]{Python/emper-without-wifi.png}
\end{figure}

Гистограммы
\begin{figure}[H]
      \centering
      \includegraphics[width=0.5\linewidth]{Python/hist-all-phones.png}
\end{figure}
\begin{figure}[H]
      \centering
      \includegraphics[width=0.5\linewidth]{Python/hist-wi-fi.png}
\end{figure}
\begin{figure}[H]
      \centering
      \includegraphics[width=0.5\linewidth]{Python/hist-without-wi-fi.png}
\end{figure}

Box-plot:
\begin{figure}[H]
      \centering
      \includegraphics[width=0.5\linewidth]{Python/box-plot.png}
\end{figure}

\subsection{Код}
\begin{verbatim}
      import pandas as pd
import matplotlib.pyplot as plt

# Загрузка данных из файла
df = pd.read_csv('mobile_phones.csv', delimiter=",", skiprows=1, names=['battery_power','blue','clock_speed','dual_sim',
                                                            'fc','four_g','int_memory','m_dep','mobile_wt','n_cores',
                                                            'pc','px_height','px_width','ram','sc_h','sc_w','talk_time',
                                                            'three_g','touch_screen','wifi','price_range'])

print(df[['dual_sim', 'n_cores', 'three_g']])

dual_sim_cnt = sum(df['dual_sim'])
three_g_cnt = sum(df['three_g'] == 1)
max_cores = max(df['n_cores'])

print("Количество моделей с двумя симкартами:", dual_sim_cnt)
print("Количество моделей поддерживающих 3g:", three_g_cnt)
print("Наибольшее число ядер у процессора:", max_cores)

srednee_battery_power = df['battery_power'].mean()
dispersia_battery_power = df['battery_power'].var()
mediana_battery_power = df['battery_power'].median()
kvantil_battery_power = df['battery_power'].quantile(2 / 5)

print("Выборочное среднее емкости аккумулятора:", srednee_battery_power)
print("Выборочная дисперсия емкости аккумулятора:", dispersia_battery_power)
print("Выборочная медиана емкости аккумулятора:", mediana_battery_power)
print("Выборочная квантиль порядка 2/5 емкости аккумулятора:", kvantil_battery_power)


with_wifi = df[df['wifi'] == 1]['battery_power']
without_wifi = df[df['wifi'] == 0]['battery_power']

# empericheskaya func
plt.figure(figsize=(10, 5))
plt.hist(df['battery_power'], bins=30, density=True, cumulative=True, histtype='step', label='Все телефоны', color='blue')
plt.xlabel('Емкость аккумулятора')
plt.ylabel('Эмпирическая функция распределения')
plt.title('График эмпирической функции распределения емкости аккумулятора')
plt.legend()
plt.grid(True)
plt.savefig('emper-all-phones.png')
plt.show()

plt.figure(figsize=(10, 5))
plt.hist(with_wifi, bins=30, density=True, cumulative=True, histtype='step', label='С Wi-Fi', color='green')
plt.xlabel('Емкость аккумулятора')
plt.ylabel('Эмпирическая функция распределения')
plt.title('График эмпирической функции распределения емкости аккумулятора')
plt.legend()
plt.grid(True)
plt.savefig('emper-wi-fi.png')
plt.show()

plt.figure(figsize=(10, 5))
plt.hist(without_wifi, bins=30, density=True, cumulative=True, histtype='step', label='Без Wi-Fi', color='red')
plt.xlabel('Емкость аккумулятора')
plt.ylabel('Эмпирическая функция распределения')
plt.title('График эмпирической функции распределения емкости аккумулятора')
plt.legend()
plt.grid(True)
plt.savefig('emper-without-wifi.png')
plt.show()

# gistogramma
plt.figure(figsize=(10, 5))
plt.hist(df['battery_power'], bins=30, alpha=0.7, label='Все телефоны', color='blue')
plt.xlabel('Емкость аккумулятора')
plt.ylabel('Частота')
plt.title('Гистограмма емкости аккумулятора')
plt.legend()
plt.grid(True)
plt.savefig('hist-all-phones.png')
plt.show()

plt.figure(figsize=(10, 5))
plt.hist(with_wifi, bins=30, alpha=0.7, label='С Wi-Fi', color='green')
plt.xlabel('Емкость аккумулятора')
plt.ylabel('Частота')
plt.title('Гистограмма емкости аккумулятора')
plt.legend()
plt.grid(True)
plt.savefig('hist-wi-fi.png')
plt.show()

plt.figure(figsize=(10, 5))
plt.hist(without_wifi, bins=30, alpha=0.7, label='Без Wi-Fi', color='red')
plt.xlabel('Емкость аккумулятора')
plt.ylabel('Частота')
plt.title('Гистограмма емкости аккумулятора')
plt.legend()
plt.grid(True)
plt.savefig('hist-without-wi-fi.png')
plt.show()

# box plot
plt.figure(figsize=(10, 5))
plt.boxplot([df['battery_power'], with_wifi, without_wifi], labels=['Все телефоны', 'С Wi-Fi', 'Без Wi-Fi'])
plt.ylabel('Емкость аккумулятора')
plt.title('Box-plot для емкости аккумулятора')
plt.grid(True)
plt.savefig('box-plot.png')
plt.show()

\end{verbatim}


\end{document}